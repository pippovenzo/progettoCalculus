
\documentclass[a4paper, 10pt, colorlinks=true, linkcolor=black, urlcolor=blue, citecolor=blue, hidelinks]{article}
\usepackage{xcolor}

\usepackage[a-1b]{pdfx}

\usepackage[italian]{babel} % Language

\usepackage{comment}      
\usepackage{graphicx} % Required for inserting images
\usepackage{multirow}
\usepackage{minted} 

\usepackage{graphicx}
\usepackage{geometry}
\usepackage{csquotes}

\usepackage{float} % Required for the [H] option
\usepackage{caption}
%\captionsetup[figure]{font=small,labelfont=small}

\usepackage{setspace}

% --- CONFIGURAZIONE HYPERREF (Sempre per ultimo) ---
\usepackage{hyperref}
\hypersetup{
    breaklinks=true,            % Permette ai link di andare a capo
}
% ---------------------------------------------------

\begin{document}
\pagenumbering{roman}

\begin{titlepage} 
      \begin{center}
            \thispagestyle{empty}
            \begin{spacing}{2.0}
                \Large{\textbf{UNIVERSITÀ DEGLI STUDI DI PADOVA}\\DIPARTIMENTO DI MATEMATICA\\Laurea Triennale in Informatica\\}
            \end{spacing}
            \includegraphics[width=5.5cm]{Resources/UNIPD_Logo.png}
            \begin{spacing}{1.2}
                \Large{\textbf{Relazione Progetto Calcolo Numerico}}\\
            \end{spacing}

            \begin{spacing}{1.0}
                \Large{{Progetto 4. Approssimazione di dati reali su\\ Nathan's Hot Dog Eating Contest }}\\
            \end{spacing}
        
            \vfill
            \begin{spacing}{1.0} 
            \vspace{1.0cm}
            \raggedleft{\large{\textit{Marco Barbiero (2101049)}, \textit{Filippo Venzo (2113705)},\\ \textit{Simone Zecchinato (2113189)}}}
            \end{spacing}
            \vspace{2.0cm}
                \large{ANNO ACCADEMICO 2025/26}
            
        \end{center}
    \end{titlepage}
\newpage

\tableofcontents % Table of contents

\pagenumbering{arabic}

\newpage

\section{Introduzione}
Il progetto 4 del corso di Calcolo Numerico richiedeva di analizzare una fonte di dati da noi scelta e tentare il \textit{fitting} ai minimi quadrati lineari nei parametri.

\section{Dataset}
Il dataset da noi scelto fa riferimento al \href{https://nathansfranks.sfdbrands.com/en-us/promotions/hot-dog-eating-contest/}{\textit{Nathan's Hot Dog Eating Contest}}.\\
Il dataset utilizzato può essere trovato al seguente link \url{https://www.kaggle.com/datasets/maraglobosky/hot-dog-eating-contest-results/data}.
\\

Di tale dataset, i parametri da noi analizzati sono stati gli anni e il numero di hot dog mangiati dal vincitore della competizione in quel particolare anno.

E' importante notare come la competizione, a partire dal 2011, sia stata divisa in due categorie: maschile e femminile, mentre prima di quell'anno era tenuta in una sola categoria aperta a tutti. 
La nostra analisi prende in considerazione, dal 2011 in poi, i vincitori della categoria maschile.

\begin{figure}[H]
    \centering
    \includegraphics[width=1\textwidth]{./Resources/GraficoDati.pdf}
    \caption{Dati grezzi}
    \label{fig:mpc1-mpc2}
\end{figure}

\section{Modelli proposti}

\subsection{Modello 1: Lineare}

Il primo modello con cui abbiamo tentato il fitting dei dati è stato un modello lineare del tipo:
\[y = \beta_0  + \beta_1 t\]

\begin{figure}[H]
    \centering
    \includegraphics[width=1\textwidth]{./Resources/linearmodel.pdf}
    \caption{Grafico fitting modello lineare $R^2 = 0,892$}
\end{figure}

Non cattura il punto di flesso tra 2000 e 2001

\subsection{Modello 2: Quadratico}

Il secondo modello con cui abbiamo tentato il fitting dei dati è stato un modello polinomiale quadratico del tipo: 
\[y = \beta_0  + \beta_1 t + \beta_2 t^2\]

\begin{figure}[H]
    \centering
    \includegraphics[width=1\textwidth]{./Resources/quadraticmodel.pdf}
    \caption{Grafico fitting modello lineare $R^2 = 0,897$}
\end{figure}

Il fitting del modello quadratico è praticamente uguale al quello del modello lineare è buono ma non soddisfacente, non riesce a catturare il punto di flesso tra il 2000 e il 2001, per questo penso serva un modello cubico.

\subsection{Modello 3: Cubico}




\appendix



\end{document}
