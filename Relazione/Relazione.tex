
\documentclass[a4paper, 10pt, colorlinks=true, linkcolor=black, urlcolor=blue, citecolor=blue, hidelinks]{article}
\usepackage{xcolor}

\usepackage[a-1b]{pdfx}

\usepackage[italian]{babel} % Language

\usepackage{comment}      
\usepackage{graphicx} % Required for inserting images
\usepackage{multirow}
\usepackage{minted} 
\bibliographystyle{unsrt}

\usepackage{graphicx}
\usepackage{geometry}
\usepackage{csquotes}

\usepackage{float} % Required for the [H] option
\usepackage{caption}
%\captionsetup[figure]{font=small,labelfont=small}

\usepackage{setspace}

% --- CONFIGURAZIONE HYPERREF (Sempre per ultimo) ---
\usepackage{hyperref}
\hypersetup{
    breaklinks=true,            % Permette ai link di andare a capo
}
% ---------------------------------------------------

\begin{document}
\pagenumbering{roman}

\begin{titlepage} 
      \begin{center}
            \thispagestyle{empty}
            \begin{spacing}{2.0}
                \Large{\textbf{UNIVERSITÀ DEGLI STUDI DI PADOVA}\\DIPARTIMENTO DI MATEMATICA\\Laurea Triennale in Informatica\\}
            \end{spacing}
            \includegraphics[width=5.5cm]{Resources/UNIPD_Logo.png}
            \begin{spacing}{1.2}
                \Large{\textbf{Relazione Progetto Calcolo Numerico}}\\
            \end{spacing}

            \begin{spacing}{1.0}
                \Large{{Progetto 4. Approssimazione di dati reali su\\ Nathan's Hot Dog Eating Contest }}\\
            \end{spacing}
        
            \vfill
            \begin{spacing}{1.0} 
            \vspace{1.0cm}
            \raggedleft{\large{\textit{Marco Barbiero (2101049)}, \textit{Filippo Venzo (2113705)},\\ \textit{Simone Zecchinato (2113189)}}}
            \end{spacing}
            \vspace{2.0cm}
                \large{ANNO ACCADEMICO 2025/26}
            
        \end{center}
    \end{titlepage}
\newpage

\tableofcontents % Table of contents

\pagenumbering{arabic}

\newpage

\section{Introduzione}
Il progetto 4 del corso di Calcolo Numerico richiedeva di analizzare una fonte di dati da noi scelta e tentare il \textit{fitting} ai minimi quadrati lineari nei parametri.

\section{Dataset}
Il dataset da noi scelto fa riferimento al \href{https://nathansfranks.sfdbrands.com/en-us/promotions/hot-dog-eating-contest/}{\textit{Nathan's Hot Dog Eating Contest}}.\\
Il dataset utilizzato può essere trovato al seguente link \url{https://www.kaggle.com/datasets/maraglobosky/hot-dog-eating-contest-results/data}.
\\

Di tale dataset, i parametri da noi analizzati sono stati gli anni e il numero di hot dog mangiati dal vincitore della competizione in quel particolare anno.

E' importante notare come la competizione, a partire dal 2011, sia stata divisa in due categorie: maschile e femminile, mentre prima di quell'anno era tenuta in una sola categoria aperta a tutti. 
La nostra analisi prende in considerazione, dal 2011 in poi, i vincitori della categoria maschile.\\

Il dataset disponibile su \textit{Kaggle} non è aggiornato al 2025, per questo è stato aggiornato da noi in una versione locale con i dati disponibili nel sito dell'oraganizzatore del contest.

\begin{figure}[H]
    \centering
    \includegraphics[width=1\textwidth]{./Resources/GraficoDati.pdf}
    \caption{Dati grezzi}
    \label{fig:mpc1-mpc2}
\end{figure}

\subsection{Analisi dei dati}
Il dataset da noi considerato è stato analizzati in un paper, \textit{Smoliga, 2020} \cite{Smoliga:2020}, che stima la massima capacità gastrica umana sulla base dei dati, in termini di \textit{Active Consumption Rate} (ACR) ovvero la massa di cibo consumata in un determinato periodo di tempo. 

Nel caso del \textit{Nathan's Hot Dog Eating Contest} il cibo considerato è un Hot Dog, compreso di pane (\textit{Bun}) e salsiccia (\textit{hot dog}), con massa di circa 100g. 

L'analisi fatta da \textit{Smoliga, 2020} conclude che la massima ACR teorica, per individui con plasticità dell'apparato digerente allenata, è di 832 g/min$^{-1}$ che corrisponde a 83.2 Hot Dog in 10 minuti.

Da questo risultato si conclude che l'apparto digerente umano, seppur allenato, possiede quindi un limite fisico che non permetterà di superare il record di circa 83 Hot Dog una volta raggiunto. 


\section{Modelli proposti}

\subsection{Modello 1: Lineare}

Il primo modello con cui abbiamo tentato il fitting dei dati è stato un modello lineare del tipo:
\[y = \beta_0  + \beta_1 t\]

\begin{figure}[H]
    \centering
    \includegraphics[width=1\textwidth]{./Resources/linearmodel.pdf}
    \caption{Grafico fitting modello lineare $R^2 = 0,892$}
\end{figure}

Non cattura il punto di flesso tra 2000 e 2001

\subsection{Modello 2: Quadratico}

Il secondo modello con cui abbiamo tentato il fitting dei dati è stato un modello polinomiale quadratico del tipo: 
\[y = \beta_0  + \beta_1 t + \beta_2 t^2\]

\begin{figure}[H]
    \centering
    \includegraphics[width=1\textwidth]{./Resources/quadraticmodel.pdf}
    \caption{Grafico fitting modello lineare $R^2 = 0,897$}
\end{figure}

Il fitting del modello quadratico è praticamente uguale al quello del modello lineare è buono ma non soddisfacente, non riesce a catturare il punto di flesso tra il 2000 e il 2001, per questo penso serva un modello cubico.

\subsection{Modello 3: Cubico}

Il terzo modello con cui abbiamo tentato il fitting dei dati è stato un modello polinomiale cubico del tipo: 
\[y = \beta_0  + \beta_1 t + \beta_2 t^2 + \beta_3 t^3\]

\begin{figure}[H]
    \centering
    \includegraphics[width=1\textwidth]{./Resources/cubicmodel.pdf}
    \caption{Grafico fitting modello lineare $R^2 = 0,948$}
\end{figure}

Possiamo notare che il modello cubico, rispetto ai due precedenti, è di gran lunga migliore, con fit maggiore di 0,1







\bibliography{citations}
\end{document}

